\documentclass[10pt,oneside,a4paper]{article}

\usepackage[latin1]{inputenc} 
\usepackage[italian]{babel}
\usepackage{siunitx} %Inserisce automaticamente i dati con le unit� di misura correttamente formattate del SI (utilizzo: \SI{0.82}{m^2}, in generale \SI{misura con il punto decimale}{unit� di misura})
\usepackage{listings} %Per citare codice informatico formattandolo correttamente
\usepackage{amsmath}
\usepackage{graphicx}
\usepackage{epigraph}

\setcounter{section}{-1}

\title{\textsc{Misura della \emph{densit�}.}}
\author{\small{G. Galbato Muscio} \and \small{L. Gravina} \and \small{L. Graziotto} \and \small{M. Rescigno}}
\date{}

\begin{document}
\begin{figure}
	\centering
	\includegraphics[scale=0.5,trim={2.8cm 8.9cm 0 9cm},clip]{logo.png}
\end{figure}
\maketitle
\begin{center} 
\fbox{{\fontsize{13pt}{8mm}\textsc{Gruppo B2.3}}} \\
\vspace{1cm}
\begin{tabular}{ccc}
	 Esperienza di laboratorio && Consegna della relazione \\
	  \emph{\small{27 marzo 2017}} && \emph{\small{3 aprile 2017}} \\
\end{tabular} 

\vspace{0.5cm}

\end{center}
\hrule
\vspace{0.5cm}
\begin{abstract}
\[
	\rho = \frac{m}{V} \qquad [\rho] = \SI{}{Kg/m^3}
\]
La densit� � bellissima e poco conosciuta per cui cercheremo qui e ora di descriverne un metodo di determinazione olistico.
\end{abstract}
\vspace{0.5cm}
\tableofcontents %Indice
\pagebreak
\section{Convenzioni}
Qui introdurremo le convenzioni usate. Sappiate che per quanto riguarda l'arrotondamento useremo la mia (Gabriele) convenzione personale in quanto sono arrivato per primo: arrotonderemo $x,5$ in base a ci� che verr� dopo il $5$, cio� per esempio 
\[
	4,351 \rightarrow 4,3 \qquad 4,356 \rightarrow 4,4.
\]
Consiglio inoltre di utilizzare la \emph{comma} (",") come separatore decimale in luogo del punto per almeno due motivi:
\begin{enumerate}
	\item siamo italiani e dobbiamo preservare le nostre tradizioni invece di adattarci a quelle degli aglosassoni (per ricordarvelo, quelli l� usano i \emph{galloni} per misurare i volumi),
	\item il pacchetto \emph{siunitx} che gestisce le unit� di misura secondo il SI utilizza la comma come separatore decimale.
\end{enumerate}
\subsection{Convenzioni particolari}
In realt� � difficile che adotteremo convenzioni non generali ma volevo testare le capacit� dell'indice.

\section{Scopo e descrizione dell'esperienza}

\end{document}